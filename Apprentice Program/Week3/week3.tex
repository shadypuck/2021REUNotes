\documentclass[../apprentice.tex]{subfiles}

\pagestyle{main}
\renewcommand{\chaptermark}[1]{\markboth{\chaptername\ \thechapter}{}}
\setcounter{chapter}{2}

\begin{document}




\chapter{}
\section{PSet 4}
\begin{enumerate}
    \item \marginnote{7/7:}Consider two lines in the plane with the angle $\gamma$ between them and suppose a grasshopper is jumping from one line to the other. Every jump is exactly 30 inches long, and the grasshopper jumps backwards whenever it has no other options. Prove that the sequence of its jumps is periodic if and only if $\frac{\gamma}{\pi}$ is a rational number.
    \begin{proof}
        Two reflections are a rotation. Think about projective geometry. Two jumps is isomorphic to a rotation by $2\gamma$. If $\pi\mid\gamma$ rationally, then we can eventually rotate around the unit circle enough to get back to where we started. Pull in rigorous group theory definition of an orbit?
    \end{proof}
    \item Let $ABCD$ be a convex 4-gon and consider four squares constructed on the outside of each of its edges. Prove that the segments connecting the centers of the opposite squares are mutually perpendicular and equal in length.
    \begin{proof}
        \begin{figure}[h!]
            \centering
            \begin{tikzpicture}[
                point/.style={circle,fill,inner sep=1.5pt}
            ]
                \footnotesize
                \draw [->,help lines] (-5,0) -- (5,0) node[right]{$\R$};
                \draw [->,help lines] (0,-5) -- (0,5) node[above]{$\mathbb{I}$};
        
                \draw [semithick]
                    (1.6,-2) coordinate (1) coordinate (5) node[point,label={below right:$A$}]{}
                    -- node[pos=0.25,right]{$a$} node[pos=0.75,right]{$a$} (1,1.5) coordinate (2) node[point,label={above right:$B$}]{}
                    -- node[pos=0.25,above]{$b$} node[pos=0.75,above]{$b$} (-0.8,1.9) coordinate (3) node[point,label={above left:$C$}]{}
                    -- node[pos=0.25,left]{$c$} node[pos=0.75,left]{$c$} (-1.4,-1.2) coordinate (4) node[point,label={below left:$D$}]{}
                    -- node[pos=0.25,below]{$d$} node[pos=0.75,below]{$d$} cycle
                ;
        
                \foreach \n [remember=\n as \nm (initially 1)] in {2,3,4,5} {
                    \draw [semithick,dashed] (\n) -- ($(\n)!1!90:(\nm)$) -- ($(\nm)!1!-90:(\n)$) -- (\nm);
                }
        
                \draw ($(1)!0.5!(2)$) coordinate (m1) -- node[below]{$a$} ($(m1)!1!-90:(2)$) node(p)[point]{} node[above right]{$p$};
                \draw ($(2)!0.5!(3)$) coordinate (m2) -- node[above left,fill=white,inner sep=1.5pt]{$b$} ($(m2)!1!-90:(3)$) node(q)[point]{} node[above right]{$q$};
                \draw ($(3)!0.5!(4)$) coordinate (m3) -- node[below]{$c$} ($(m3)!1!-90:(4)$) node(r)[point]{} node[above left]{$r$};
                \draw ($(4)!0.5!(5)$) coordinate (m4) -- node[below right,fill=white,inner sep=1.5pt]{$d$} ($(m4)!1!-90:(5)$) node(s)[point]{} node[below left]{$s$};
        
                \draw [rex,thick,-stealth] (q) -- node[right]{$A$} (s);
                \draw [rex,thick,-stealth] (p) -- node[pos=0.6,above]{$B$} (r);
            \end{tikzpicture}
            \caption{Complex polygon.}
            \label{fig:complexPolygon}
        \end{figure}
        Facts:
        \begin{enumerate}
            \item If $n$ complex numbers add to 0, then the lines from the origin to them in the complex plane form a closed $n$-gon.
            \item Multiplication by $i$ is a $\ang{90}$ rotation.
        \end{enumerate}
        Let's begin. First, we define the points $p,q,r,s$ at the center of each square in terms of the "complex numbers" from the origin at $A$:
        \begin{align*}
            p &= (1-i)a&
            q &= 2a+(1-i)b&
            r &= 2a+2b+(1-i)c&
            s &= 2a+2b+2c+(1-i)d
        \end{align*}
        Let $A=\overrightarrow{qs}$, $B=\overrightarrow{pr}$. Then
        \begin{align*}
            A &= s-q&
                B &= r-p\\
            &= (1+i)b+2c+(1-i)d&
                &= (1+i)a+2b+(1-i)c
        \end{align*}
        To prove that the magnitudes of $A$ and $B$ are the same and that they are perpendicular, Fact (a) tells us that it will suffice to show that $iB=A$, or that $A-iB=0$. But we have that
        \begin{align*}
            A-iB &= (1+i)b+2c+(1-i)d-i((1+i)a+2b+(1-i)c)\\
            &= (1+i)b+2c+(1-i)d-((i-1)a+2bi+(i+1)c)\\
            &= (1-i)a+(1-i)b+(1-i)c+(1-i)d\\
            &= \frac{1-i}{2}\cdot(2a+2b+2c+2d)\\
            &= \frac{1-i}{2}\cdot 0\tag*{Fact (b)}\\
            &= 0
        \end{align*}
        as desired.
    \end{proof}
    \item Prove that a composition of three symmetries is a sliding symmetry.
    \begin{proof}
        Each symmetry reverses the orientation of a shape. Thus, three symmetries reverse the orientation like one sliding symmetry. Symmetries also have the ability to rotate an object to any desired angle (by rotating the line of symmetry). Finally, distance from the shape to the line controls position in one orthogonal direction, and the slide can control distance in the other orthogonal direction.
    \end{proof}
    \item The points $A_1,\dots,A_n$ form a regular polygon, inscribed in a circle with the center $O$. A point $X$ lies on the same circle. Prove that the images of the point $X$ under the symmetries with axes $OA_1,OA_2,\dots,OA_n$ form a regular polygon.
    \item Remove a corner from a $101\times 101$ chessboard. Prove that the rest cannot be covered by triominoes. A triomino is like a domino except that it consists of three squares in a row; each cell can cover one cell on a chessboard. Each triomino can either "stand" or "lie."
    \item Consider a finite collection of segments on a line so that every two of them intersect. Prove that all segments have a common point.
    \begin{proof}
        Let $\ell_1,\dots,\ell_n$ be a finite collection of segments on a line (i.e., a closed interval on $\R$). By the hypothesis, $\ell_i\cap\ell_j\neq\emptyset$ for all $i,j\in[n]$. We induct on $n$. For the base case $n=1$, we clearly have $\bigcap_{i\in[1]}\ell_i=\ell_1\neq\emptyset$ by the definition of a closed interval. Now suppose that we have proven that $\bigcap_{i\in[n]}\ell_i\neq\emptyset$; we seek to prove the claim for $n+1$. First off, note that $\bigcap_{i\in[n]}\ell_i$ is a closed interval in its own right, and that its lower and upper bounds are the supremum of the lower bounds of all $\ell_i$ and the infimum of the upper bounds of all $\ell_i$, respectively. Thus, to show that $\bigcap_{i\in[n+1]}\ell_i$ is nonempty, it will suffice to show that the the lower bound of $\ell_{n+1}$ is less than or equal to the upper bound of $\bigcap_{i\in[n+1]}\ell_i$, or vice versa. But clearly it must be, or $\ell_{n+1}$ would have an empty intersection with an $\ell_i$, a contradiction.
    \end{proof}
    \item Let $S$ be a set of $n+1$ integers from 1 to $2n$. Prove that at least two elements in $S$ are coprime.
\end{enumerate}




\end{document}