\documentclass[../main.tex]{subfiles}

\pagestyle{main}
\renewcommand{\chaptermark}[1]{\markboth{\chaptername\ \thechapter}{}}

\begin{document}




\chapter{}
\section{Introduction to the Program (May / Rudenko)}
\begin{itemize}
    \item \marginnote{6/21:}Mainly given by Peter May.
    \item A far broader range of mathematics than any other REU.
    \item Things you have to do:
    \begin{enumerate}
        \item Soak up as much mathematics as you can.
        \item Work with a mentor to write a paper.
        \begin{itemize}
            \item You can work with people to write a joint paper?
            \item This is fairly unique to this REU.
        \end{itemize}
        \item Meet with your mentors at least twice a week.
    \end{enumerate}
    \item Don't be shy and unwilling to ask questions.
    \item Daniil Rudenko is in charge of the apprentice program.
    \item Apprentice program:
    \begin{itemize}
        \item An opportunity particularly early in one's mathematical career to explore mathematics.
        \item Asynchronous video lectures.
        \begin{itemize}
            \item Feel free to share with friends.
        \end{itemize}
        \item Problem solving.
        \begin{itemize}
            \item Problems that are not merely exercises but more difficult, interesting processes.
            \item Spend a couple hours a day thinking about these problems.
        \end{itemize}
    \end{itemize}
    \item Emphasis on relations between different subjects.
    \item They will be organizing social activities.
    \item Social meet and greet at 6:00 PM tonight.
    \item Breakout rooms:
    \begin{enumerate}
        \item Apprentice Program.
        \item Probability.
        \item Analysis and Dynamical Systems.
        \item Algebraic Topics.
        \item Main room: Algebraic Topology.
    \end{enumerate}
    \item More on the apprentice program:
    \begin{itemize}
        \item Daniil wants us to see much more than classical analysis/calculus. He doesn't see dividing lines between fields of mathematics.
        \item Bijections, binomial coefficients, Catalan numbers, etc. to start.
        \item Group of permutations, group of isometries of the plane, what a group is, etc.
        \item We can solve problems individually or in groups.
        \begin{itemize}
            \item Some problems will say not to collaborate.
        \end{itemize}
        \item Don't try to solve every problem. Don't try to solve everything fast; it's fine if you fail, if you just think about something for a couple hours that's interesting and don't get everywhere.
        \item On campus classes option for participants in Chicago.
        \item This week 10-11 AM Wed/Fri?
        \item Office hours 10-11 AM on Thursday.
        \item He will send an email with more information.
        \item Be consistent in whether you want to be on or off campus.
        \item You may attend whatever you want, but be careful: The apprentice program is your priority, so don't spend too much time on the other stuff.
        \begin{itemize}
            \item Follow Piazza groups to get links.
        \end{itemize}
        \item \LaTeX\ one solution each week.
    \end{itemize}
\end{itemize}



\section{Introduction to Complex Dynamics 1 (Calegari)}
\begin{itemize}
    \item Main focus: the Mandelbrot Set.
    \item Let $f_c:\C\to\C$ be the quadratic polynomial $f_c(z):=z^2+c$ where $c\in\C$ is a constant and $z\in\C$ is a variable.
    \begin{itemize}
        \item We study quadratics because they're the simplest nontrivial polynomial, i.e., one that displays the interesting phenomena of higher degree polynomials.
    \end{itemize}
    \item We want to understand the dynamics of $f_c$, i.e., what happens as we apply $f_c$ over and over again.
    \begin{itemize}
        \item In other words $z\to z^2+c\to (z^2+c)^2+c\to ((z^2+c)^2+c)^2+c\to\cdots$.
        \item Are there any special values of $z$ that have interesting characteristics?
    \end{itemize}
    \item \textbf{Fixed point}: A value $z$ such that $f_c(z)=z$.
    \begin{itemize}
        \item Fixed points of $f_c$ are equivalent to \textbf{roots} of $f_c-z$.
    \end{itemize}
    \item In this branch of mathematics, we don't care so much about factoring $f_c$ as much as we care about other special entities like fixed points and \textbf{critical points}.
    \item \textbf{Critical point}: A point where $\dv*{f_c}{z}=0$.
    \item We denote $z$ large by $|z|>>1$.
    \item Note that $z^2+c$ doesn't change the magnitude of $z$ that much unless $z$ is large.
    \begin{itemize}
        \item Essentially, if $|z|>>1$, then $|f_c(z)|>>|z|$.
    \end{itemize}
    \item Introduces composition notation: $z\to f_c(z)\to {f_c}^2(z)\to {f_c}^3(z)\to\cdots$\footnote{Sometimes, people also use a circled number in the superscript.}.
    \item If $z$ large, then the sequence $z,f_c(z),{f_c}^2(z),\dots$ converges to infinity.
    \item \textbf{Riemann Sphere}: The set $\hat{\C}:=\C\cup\infty$.
    \begin{itemize}
        \item Like an open set of complex numbers.
        \item In this case, we can think of infinity as a fixed point.
    \end{itemize}
    \item Any number whose absolute value is sufficiently big will converge to infinity.
    \item Introduces big $N$ convergence test.
    \item Infinity is an \textbf{attracting fixed point}, i.e. there exists an open neighborhood $U$ containing $\infty$ such that for all $z\in U$, ${f_c}^n(z)\to\infty$ as $n\to\infty$.
    \item \textbf{Filled Julia set}: The set $\{z:\text{the iterates }{f_c}^n(z)\text{ do not converge to }\infty\}$. \emph{Also known as} $\bm{K(f_c)}$.
    \begin{itemize}
        \item Equivalent to the set $\{z:\exists\text{ a constant }T\text{ s.t. }|{f_c}^n(z)|\leq T\ \forall\ n\}$.
    \end{itemize}
    \item The points that diverge to infinity are not that interesting; their divergence is their only property.
    \item Much more interesting are the points that do not diverge to infinity.
    \item Lemma: $K(f_c)$ is closed and bounded (i.e., compact).
    \begin{itemize}
        \item Proof: There exists $T$ (depending on $c$) such that if $|z|>T$, then $z\notin K(f_c)$. Furthermore, $z\in K(f_c)$ if and only if there exists $n$ such that $|{f_c}^n(z)|>T$. Let $U:=\{z:|z|>T\}$. This is open. Thus, $z\in K(f_c)\Longleftrightarrow z$ iterates ${f_c}^n(z)\in U$. Therefore, $K(f_c)=\C$, so $\bigcup_n{f_c}^n(U)$, i.e., $K(f_c)$ is closed.
        \item Bounded because numbers are not arbitrarily large. \emph{flesh out details?}
    \end{itemize}
    \item Calegari's proofs will be somewhat informal throughout the week; he hits the main points and leaves the details as an exercise to the student.
    \item Question: What other topological properties does the filled Julia set have?
    \begin{itemize}
        \item Is it possible that $K(f_c)=\emptyset$?
        \begin{itemize}
            \item No, it is not --- as a degree 2 polynomial, $f_c-z$ has at least one root, which will by necessity be a fixed point, i.e., not diverge to infinity, i.e., in the filled Julia set.
        \end{itemize}
        \item Could it be a finite set?
        \begin{itemize}
            \item No --- $K(f_c)$ is a \textbf{perfect set}.
            \item Uncountably infinite, too.
        \end{itemize}
        \item Is $K(f_c)$ connected?
        \begin{itemize}
            \item Sometimes.
        \end{itemize}
    \end{itemize}
    \item \textbf{Perfect set}: A set where every point in the set is a \textbf{nontrivial limit point} of the set.
    \begin{itemize}
        \item Example: A closed interval, \emph{others listed}.
    \end{itemize}
    \item \textbf{Nontrivial limit point}: A point $p$ in a set $A$ such that there is a nontrivial sequence (i.e., not a constant sequence, e.g., $p,p,p,\dots$) of points in $A$ that converge to $p$.
    \item \textbf{Not connected}: A set $X\subset\C$ such that there exist disjoint, open sets $U,V$ such that $X\subset U\cup V$, $X\cap U\neq\emptyset$, and $X\cap V\neq\emptyset$.
    \item \textbf{Mandlebrot set}: The set of complex numbers $c\in\C$ such that $K(f_c)$ is connected \emph{Also known as} $\bm{M}$.
    \item We can prove that $K(f_c)$ is connected if and only if the critical point of $f_c$ is an element of $K(f_c)$.
    \begin{itemize}
        \item Remember that critical points of $f_c$ are equivalent to zeroes of $f_c'$.
    \end{itemize}
    \item Note that critical points of $f_c:=z^2+c$ are equal to the roots of $f_c'=2z$, i.e., the elements of $\{0\}$.
    \item $K(f_c)$ is connected is equivalent to the sequence $0\to c\to c^2+c\to (c^2+c)^2+c\to\cdots$ is bounded (an absolute value).
    \begin{itemize}
        \item Thus, $c\in M$ is equivalent to the sequence $0\to c\to c^2+c\to (c^2+c)^2+c\to\cdots$ is bounded.
    \end{itemize}
    \item The Mandelbrot set is compact, too.
    \item Proposition: $K(f_c)$ is connected if and only if $0\in K(f_c)$.
    \begin{itemize}
        \item "Proof": $\C-K(f_c)=\bigcup_n{f_c}^{-n}(U)$ where $U$ is an open neighborhood of $\infty$, i.e., the set $\{z:|z|>T\}$.
        \item Let $X_n:=\C-{f_c}^{-n}(U)$, i.e., $X_0=\C-U$, so $K(f_c)=\bigcap_nX_n$.
        \item Cyclic map? $X_n$ getting "smaller" as $n$ increases? $X_{n+1}\subset X_n$.
        \item Assume $X_n=\text{little}$.
        \item Two cases: $X_n$ contains 0 and $X_n$ does not contain 0.
        \item Either every preimage of $X_n$ is connected or there is a $T$ such that for all $n\geq T$, $X_n$ is not connected.
    \end{itemize}
    \item Theorem (Douady-Hubbard): $M$ is connected.
\end{itemize}



\section{Harmonic Functions, Brownian Motion, and Analysis in the Plane 1 (Lawler)}
\begin{itemize}
    \item These topics will change week to week, so drop in at any point over the summer.
    \item Schedule:
    \begin{itemize}
        \item Lectures MWF at 2:30 PM.
        \item Group meeting Tuesday at 2:30 PM.
        \begin{itemize}
            \item Anybody can attend these!
        \end{itemize}
        \item No Zoom on Thursday, but there will be an opportunity to talk to Greg Lawler in person at the department of mathematics outside Eckhart when the weather is good.
    \end{itemize}
    \item Resources:
    \begin{itemize}
        \item Piazza --- look under the resources tab for lecture notes (with some exercises; these are very rough; gives you something to read with the lectures), other materials, etc.
        \item There is a 180 page book draft based on his REU lectures last summer.
        \begin{itemize}
            \item Do not share this.
        \end{itemize}
    \end{itemize}
    \item This math is at the border of analysis (basically advanced calculus) and probability.
    \begin{itemize}
        \item Lawler thinks of these as all basically the same subject.
    \end{itemize}
    \item We will work in $\R^2$.
    \item A lot of what Dr. Lawler does is often called Complex Analysis.
    \item Complex analysis allows you to get the results quicker even though they encapsulate ideas that are 100\% real; we're going to take a real-function perspective.
    \item Harmonic function notation:
    \begin{itemize}
        \item Domains $D$ are connected open sets that are subsets of $\R^2$.
        \item Mean value: $f:\R^2\to\R$ (continuous), or $f:D\to\R$.
        \item $z,w$ are points in $\R^2$, and we write $z=(x,y)$ where $x,y$ are the one-dimensional components.
        \item $B(z,\epsilon)=\{w:|z-w|<\epsilon\}$ is an open disk and $\partial\, B(Z,\epsilon)$ is the circle of radius $\epsilon$ about $z$.
    \end{itemize}
    \item If $B(z,\epsilon)\subset D$, then the (circular) mean value $MV(f;z,\epsilon)$ is the average rate of $f$ on $\partial B(z,\epsilon)$, i.e., the quantity
    \begin{equation*}
        \frac{1}{2\pi\epsilon}\int_{\{|w-z|=\epsilon\}}f(w)|\dd{w}|
    \end{equation*}
    where $|\dd{w}|$ is with respect to arc length.
    \item Let $(\cos\theta,\sin\theta)=\e[i\theta]$.
    \item \textbf{Harmonic function}: $f:D\to\R$ is harmonic if $f$ is continuous and for all $z\in D$ and every $\epsilon>0$ with $d(z,\partial D)>\epsilon$, then $f(z)=MV(f;z,\epsilon)$.
    \item Many applications, notably in physics wrt. heat.
    \begin{itemize}
        \item Consider $D$ describing a surface with heat. Fix the temperature at the boundary. Let $U(z)=\text{temperature at }z$ (in equilibrium).
        \item Then $U$ is harmonic on $D$.
    \end{itemize}
    \item We're going to understand the mean value in terms of the \textbf{Laplacian}.
    \item If $f:D\to\R$ is $C^2$ (the first and second derivatives exist and are continuous [either two derivatives in one variable or one derivative in both variables for $\R^2$]), then the Laplacian is defined by
    \begin{equation*}
        \Delta f(z) = f_{xx}(z)+f_{yy}(z)
    \end{equation*}
    \item Proposition: If $u$ is $C^2$ in $D$, then $\Delta u(z)=\lim_{\epsilon\to 0}4\cdot\frac{MV(u;z,\epsilon)-u(z)}{\epsilon^2}$.
    \item For ease, let's assume that $z=0=(0,0)$ and $u(z)=0$.
    \item Taylor polynomial (in several variables): If $z=(x,y)$, then
    \begin{equation*}
        u(z) = 0+u_x(0)x+u_y(0)y+\frac{1}{2}u_{xx}(0)x^2+\frac{1}{2}u_{yy}(0)y^2+u_{xy}(0)xy+\sigma(|z|^2)\footnotemark
    \end{equation*}
    \footnotetext{$\sigma$ is pronounced "little oh."}
    \begin{itemize}
        \item $u_x(0)MV(x;0,\epsilon)+u_y(0)MV(y;0,\epsilon)+u_{xy}(0)MV(xy;0,\epsilon)+\sigma(\epsilon^2)+\frac{1}{2}[u_{xx}(0)x^2+u_{yy}(0)y^2]$.
        \item Note that $u_{xx}(0)x^2=MV(x^2;0,\epsilon)$ and $u_{yy}(0)y^2=MV(y^2;0,\epsilon)$.
        \item You can use multivariable calculus, or you can observe that $MV(x^2;0,\epsilon)=MV(y^2;0,\epsilon)$, thus telling you that $MV(x^2;0,\epsilon)+MV(y^2;0,\epsilon)=MV(x^2+y^2;0,\epsilon)=\epsilon^2$.
        \item Since $|z|^2=\epsilon^2$, we have that $u(z)=\frac{1}{2}[\frac{1}{2}]$...
    \end{itemize}
    \item Proposition: A function $f:D\to\R$ is harmonic if and only if it is $C^2$ and $\Delta f(z)=0$ for all $z\in D$.
    \begin{itemize}
        \item Proof: Backwards direction first. We want to show that $C^2$ and $\Delta f(z)=0$ imply the mean value property. The mean value property clearly holds at $\epsilon=0$. Consider $MV(f;z,\epsilon)$ as a function of $\epsilon$. The derivative in $\epsilon$ ends up looking something like $\frac{1}{2\pi\epsilon}\int_\text{circle}\partial_nf(w)|\dd{w}|$ where $\partial_n$ is the normal direction.
        \item Using the divergence theorem, we have that the above is equal to $\int_\text{disk}\Delta f(w)\dd{w}$. Note that we sometimes write $\Delta f=\dd{w}(\nabla f)$ where $\nabla f=(f_x,f_y)$. Additionally, $\text{div}\,(\nabla f)=\partial_x(f_x)+\partial_y(f_y)$.
        \item Exercise: Show that if $u$ is harmonic, then $u$ is $C^2$.
    \end{itemize}
    \item The notion of probability comes in when we ask, "what is the `mean value' if we are not a disk viewed from the center?"
\end{itemize}



\section{The Mathematics of Playing Pool (Mazur)}
\begin{itemize}
    \item Main focus: Billiards in a polygon.
    \item The ball bounces off a side with the same angle of incidence it struck it with. If the ball hits the corner, it stops (maybe it fell into a pocket).
    \item \textbf{Billiards}: Start with a polygon in the plane. Shoot a billiard ball, thought of as a point mass, ...
    \item Rectangular tables are fully understood, but other polygons are harder. Curved sides are even more complicated.
    \item Connection to physics: Ehrenfest windtree model (by Paul and Tatjana Ehrenfest, 1912).
    \item One thing people study is the diffusion rate of a random particle. This means that if you take a random particle and follow it for a long time $t$, how far is it from where it started? What people know is that a typical particle is about distance $t^{2/3}$ away.
    \item Another example: Take two point masses with positions $0\leq x_1\leq x_2\leq 1$ on the unit interval $[0,1]$. Suppose their masses are $m_1,m_2$ and they move with velocities $v_1,v_2$, respectively. They collide with each other and with the barriers at 0 and 1. Momentum and energy are conserved.
    \begin{itemize}
        \item We can convert this to billiards in a right triangle with the observations that energy and speed are related and momentum and angle of incidence are related.
    \end{itemize}
    \item Billiards are important examples of \textbf{dynamical systems} where one studies behavior of objects under a deterministic system (initial position and velocity define motion for the rest of time).
    \item Billiards questions:
    \begin{itemize}
        \item Are there periodic orbits?
        \item How does a typical orbit behave in the long term? Is it dense? Is it equidistributed?
        \item Illumination problem (can you get from any point to any other?).
    \end{itemize}
    \item Periodic orbits:
    \begin{itemize}
        \item There are periodic orbits in acute triangles.
        \item Drop perpendiculars; use Euclidean geometry to prove.
        \item It is unknown if a general obtuse triangle has a periodic orbit. This is considered to be one of the big unsolved problems in dynamics\footnote{Can we consider the set of all possible initial positions and directions and see what converges to what?}.
    \end{itemize}
    \item Equidistribution and Ergodicity:
    \begin{itemize}
        \item ...
    \end{itemize}
    \item Rational billiards is much more well-defined. Every rational table has periodic orbits.
    \item Most paths equidistribute.
    \item Illumination problem:
    \begin{itemize}
        \item Now imagine you put a light source at a point on your table. The walls are mirrors and a light beam bounces of the mirror with angle of incidence equal to angle of reflection. Is every point illuminated? In other words, can you get from any point to any other? Not in a Penrose unilluminable room (a region is dark in this elliptical room).
        \item Polygon example from Tokarski in the 1980s (a zero-dimensional point is unilluminable).
        \item Within the last 5 years: For any rational billiard, there are at most a finite number of unilluminable points.
    \end{itemize}
    \item Unfolding billiards boards.
    \item If the slope of the line on a torus is rational, it closes up. If the slope of the line on a torus is irrational, it does not close but is equidistributed.
    \item For a square, when we glue the unfolded version together, we get a genus 1 surface (a torus).
    \item For a triangle with angles $\frac{\pi}{2}$, $\frac{\pi}{8}$, and $\frac{3\pi}{8}$, the unfolded version can be glued together into a genus 2 surface.
    \item Ergodicity:
    \begin{itemize}
        \item A common notion in mathematics is that of irreducibility.
        \item In our context, irreducible (or ergodic) means you cannot divide your table $X$ nontrivially into 2 pieces $X=X_1\cup X_2$ so that if you start with a point in $X_1$ and you move in a straight line, you stay in $X_1$ and if you start in $X_2$ you stay there.
        \item In other words, there are no invariant sets.
    \end{itemize}
    \item Proves the Kronecker-Weyl theorem.
\end{itemize}



\section{Introduction to Complex Dynamics 2 (Calegari)}
\begin{itemize}
    \item \marginnote{6/22:}Picking up from yesterday on the proof of the proposition, $K(f_c)$ is connected iff $0\in K(f_c)$.
    \begin{itemize}
        \item Recall that 0 is the unique critical point.
        \item Also recall that $U$ is the set that contains only elements with sufficiently big absolute values, big enough so that $f_c(U)\subset U$.
        \item Define $X_0=\C-U$, $X_{n+1}={f_c}^{-n}(X_n)$.
        \begin{itemize}
            \item Each $X_n$ is compact, and $K(f_c)$ is equal to the intersection of all $X_n$, therefore compact in and of itself.
        \end{itemize}
        \item Thus, $K(f_c)$ is connected iff every $X_n$ is connected.
        \item Key point: If $0\in K(f_c)$, then each $X_n$ is a disk; otherwise, some $X_n$ is not a disk.
        \item Fact: Every point in $\C$ has exactly 2 preimages under $f_c$ except for the critical value $c=f_c(0)$ since $f_c$ is a degree 2 polynomial.
        \item Assume $X_n$ is a disk. If $X_n$ contains the critical value, then $X_{n+1}$ is a disk; otherwise, not (in fact, it will be disconnected).
    \end{itemize}
    \item Under $f_c$, the preimage of a circle not containing the critical value is either 2 circles, each of which maps one-to-one, or a single circle mapping two-to-one.
    \begin{itemize}
        \item Suppose that the preimage of the boundary of the circles is two distinct circles.
        \item By continuity, concentric circles narrowing down within the original set narrow down within the other two circles.
        \item Each point in $X_n$ has exactly two preimages iff $c\notin X_n$.
        \item As we make smaller and smaller circles, than we can split our one-circle preimage into two disconnected subsets.
    \end{itemize}
    \item Today: The theory of Julia sets for holomorphic functions in general.
    \item Let $f$ be a \textbf{holomorphic} map from $\hat{\C}$ to itself.
    \item Every such $f$ has finitely many 0s and poles ($\infty$s).
    \item Therefore, $f$ is a rational function, i.e., a ratio of polynomials. Symbolically, $f(z)=\frac{p(z)}{q(z)}$ for some polynomials $p,q$.
    \item To talk about Julia sets, we need some definitions.
    \item \textbf{Normal family}: Let $U$ be an open subset of the Riemann sphere $\hat{\C}$, and let $F$ be a family of holomorphic functions $f:U\to\hat{\C}$. $F$ is \textbf{normal} if its closure (in the space of all holomorphic functions from $U$ to $\hat{C}$) is compact.
    \begin{itemize}
        \item In other words, if ever infinite sequence $f_n\in F$ has a subsequence that converges locally uniformly to some limit $g:U\to\hat{\C}$.
    \end{itemize}
    \item Normality is local.
    \item Proposition: Suppose $F$ is a family of holomorphic functions defined on $U$, and suppose for all $p\in U$, there exists open $p\in V\subset U$ such that $F|_V$ is normal. Then $F|_U$ is normal.
    \begin{itemize}
        \item Proof 1: Diagonal argument.
        \item Proof 2: ???
    \end{itemize}
    \item \textbf{Julia set}: Let $f$ be a holomorphic map from $\hat{\C}$ to itself. Let $\mathcal{F}:=\{f^n\mid n\in\N\}$. The Fatou set $\Omega_f:\subset\hat{\C}$ is the open subset whose $\mathcal{F}$ is normal. It is equal to the union of all $U$ where $F|_U$ is normal. Thus, $\Omega_f$ is open and $\mathcal{F}|_{\Omega_f}$ is normal. The \textbf{Julia set} $J_f:\subset\hat{\C}$ is $\hat{C}-\Omega_f$, i.e., $p\in J_f$ iff for all $U$ containing $p$, $F|_U$ is not normal on $U$.
    \begin{itemize}
        \item Hence, $J_f$ is compact.
    \end{itemize}
    \item Example: Let's let $p$ be a fixed point for $f$.
    \begin{itemize}
        \item $p$ is an attracting fixed point if $|f'(p)|<1$. $p$ is super attracting if $|f'(p)|=0$.
        \item Example:
        \begin{itemize}
            \item If $f$ is a polynomial of degree at least 2, then $\infty$ is a super attracting fixed point.
        \end{itemize}
        \item If we take a sufficiently small neighborhood of $p$, then $f$ shrinks and rotates the neighborhood a little bit.
        \item \textbf{Basin of attraction} of $p$.
        \item \textbf{Immediate basin of attraction} of $p$ is the connected component of the basin of attraction of $p$.
    \end{itemize}
    \item If $f$ is a polynomial, then $K(f_c)$ is equal to the complement of the basin of infinity.
    \item "Most" "typical" $f$ have $\Omega_f=\bigcup\text{basins of attraction of attracting periodic orbits}$.
    \begin{itemize}
        \item Furthermore: every immediate basin of an attracting periodic orbit contains at least one critical point.
        \item A rational map of degree $d$ (the maximum of the degrees of polynomials $p,q$) has $2d-2$ critical points.
    \end{itemize}
    \item Theorem: The closure of the set of repelling periodic orbits (i.e., $p$ with $f^n(p)=p$ and $|(f^n)'(p)|>1$) is $J_f$.
\end{itemize}



\section{Coambiguous Concepts 1 (May)}
\begin{itemize}
    \item Categories can inscribe things of different types.
    \item Categories can be interesting mathematical objects in and of themselves much like rings, groups, etc.
    \item Whenever you define an object, you should define a notion of a morphism (or map) between objects.
    \item Morphisms have compositions and identities.
    \item A monoid is a category with one object. A category is a monoid with many objects!
    \item Today: Category theory.
    \item \textbf{Poset}: A \underline{p}artially \underline{o}rdered \underline{set}, i.e., one that is transitive, reflexive, and antisymmetric ($A\leq B\text{ and }A\geq B\Longleftrightarrow A=B$).
    \item \textbf{Small category}: A category that has a set of objects as opposed to a class of objects. \emph{Also known as} \textbf{kitty category}, \textbf{kittygory}.
\end{itemize}




\end{document}