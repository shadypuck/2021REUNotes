\documentclass[../main.tex]{subfiles}

\pagestyle{main}
\renewcommand{\chaptermark}[1]{\markboth{\chaptername\ \thechapter}{}}

\begin{document}




\chapter{}
\section{Introduction to the Program (May / Rudenko)}
\begin{itemize}
    \item \marginnote{6/21:}Mainly given by Peter May.
    \item A far broader range of mathematics than any other REU.
    \item Things you have to do:
    \begin{enumerate}
        \item Soak up as much mathematics as you can.
        \item Work with a mentor to write a paper.
        \begin{itemize}
            \item You can work with people to write a joint paper?
            \item This is fairly unique to this REU.
        \end{itemize}
        \item Meet with your mentors at least twice a week.
    \end{enumerate}
    \item Don't be shy and unwilling to ask questions.
    \item Daniil Rudenko is in charge of the apprentice program.
    \item Apprentice program:
    \begin{itemize}
        \item An opportunity particularly early in one's mathematical career to explore mathematics.
        \item Asynchronous video lectures.
        \begin{itemize}
            \item Feel free to share with friends.
        \end{itemize}
        \item Problem solving.
        \begin{itemize}
            \item Problems that are not merely exercises but more difficult, interesting processes.
            \item Spend a couple hours a day thinking about these problems.
        \end{itemize}
    \end{itemize}
    \item Emphasis on relations between different subjects.
    \item They will be organizing social activities.
    \item Social meet and greet at 6:00 PM tonight.
    \item Breakout rooms:
    \begin{enumerate}
        \item Apprentice Program.
        \item Probability.
        \item Analysis and Dynamical Systems.
        \item Algebraic Topics.
        \item Main room: Algebraic Topology.
    \end{enumerate}
    \item More on the apprentice program:
    \begin{itemize}
        \item Daniil wants us to see much more than classical analysis/calculus. He doesn't see dividing lines between fields of mathematics.
        \item Bijections, binomial coefficients, Catalan numbers, etc. to start.
        \item Group of permutations, group of isometries of the plane, what a group is, etc.
        \item We can solve problems individually or in groups.
        \begin{itemize}
            \item Some problems will say not to collaborate.
        \end{itemize}
        \item Don't try to solve every problem. Don't try to solve everything fast; it's fine if you fail, if you just think about something for a couple hours that's interesting and don't get everywhere.
        \item On campus classes option for participants in Chicago.
        \item This week 10-11 AM Wed/Fri?
        \item Office hours 10-11 AM on Thursday.
        \item He will send an email with more information.
        \item Be consistent in whether you want to be on or off campus.
        \item You may attend whatever you want, but be careful: The apprentice program is your priority, so don't spend too much time on the other stuff.
        \begin{itemize}
            \item Follow Piazza groups to get links.
        \end{itemize}
        \item \LaTeX\ one solution each week.
    \end{itemize}
\end{itemize}



\section{Introduction to Complex Dynamics (Calegari)}
\begin{itemize}
    \item Main focus: the Mandelbrot Set.
    \item Let $f_c:\C\to\C$ be the quadratic polynomial $f_c(z):=z^2+c$ where $c\in\C$ is a constant and $z\in\C$ is a variable.
    \begin{itemize}
        \item We study quadratics because they're the simplest nontrivial polynomial, i.e., one that displays the interesting phenomena of higher degree polynomials.
    \end{itemize}
    \item We want to understand the dynamics of $f_c$, i.e., what happens as we apply $f_c$ over and over again.
    \begin{itemize}
        \item In other words $z\to z^2+c\to (z^2+c)^2+c\to ((z^2+c)^2+c)^2+c\to\cdots$.
        \item Are there any special values of $z$ that have interesting characteristics?
    \end{itemize}
    \item \textbf{Fixed point}: A value $z$ such that $f_c(z)=z$.
    \begin{itemize}
        \item Fixed points of $f_c$ are equivalent to \textbf{roots} of $f_c-z$.
    \end{itemize}
    \item In this branch of mathematics, we don't care so much about factoring $f_c$ as much as we care about other special entities like fixed points and \textbf{critical points}.
    \item \textbf{Critical point}: A point where $\dv*{f_c}{z}=0$.
    \item We denote $z$ large by $|z|>>1$.
    \item Note that $z^2+c$ doesn't change the magnitude of $z$ that much unless $z$ is large.
    \begin{itemize}
        \item Essentially, if $|z|>>1$, then $|f_c(z)|>>|z|$.
    \end{itemize}
    \item Introduces composition notation: $z\to f_c(z)\to {f_c}^2(z)\to {f_c}^3(z)\to\cdots$\footnote{Sometimes, people also use a circled number in the superscript.}.
    \item If $z$ large, then the sequence $z,f_c(z),{f_c}^2(z),\dots$ converges to infinity.
    \item \textbf{Riemann Sphere}: The set $\hat{\C}:=\C\cup\infty$.
    \begin{itemize}
        \item Like an open set of complex numbers.
        \item In this case, we can think of infinity as a fixed point.
    \end{itemize}
    \item Any number whose absolute value is sufficiently big will converge to infinity.
    \item Introduces big $N$ convergence test.
    \item Infinity is an \textbf{attracting fixed point}, i.e. there exists an open neighborhood $U$ containing $\infty$ such that for all $z\in U$, ${f_c}^n(z)\to\infty$ as $n\to\infty$.
    \item \textbf{Filled Julia set}: The set $\{z:\text{the iterates }{f_c}^n(z)\text{ do not converge to }\infty\}$. \emph{Also known as} $\bm{K(f_c)}$.
    \begin{itemize}
        \item Equivalent to the set $\{z:\exists\text{ a constant }T\text{ s.t. }|{f_c}^n(z)|\leq T\ \forall\ n\}$.
    \end{itemize}
    \item The points that diverge to infinity are not that interesting; their divergence is their only property.
    \item Much more interesting are the points that do not diverge to infinity.
    \item Lemma: $K(f_c)$ is closed and bounded (i.e., compact).
    \begin{itemize}
        \item Proof: There exists $T$ (depending on $c$) such that if $|z|>T$, then $z\notin K(f_c)$. Furthermore, $z\in K(f_c)$ if and only if there exists $n$ such that $|{f_c}^n(z)|>T$. Let $U:=\{z:|z|>T\}$. This is open. Thus, $z\in K(f_c)\Longleftrightarrow z$ iterates ${f_c}^n(z)\in U$. Therefore, $K(f_c)=\C$, so $\bigcup_n{f_c}^n(U)$, i.e., $K(f_c)$ is closed.
        \item Bounded because numbers are not arbitrarily large. \emph{flesh out details?}
    \end{itemize}
    \item Calegari's proofs will be somewhat informal throughout the week; he hits the main points and leaves the details as an exercise to the student.
    \item Question: What other topological properties does the filled Julia set have?
    \begin{itemize}
        \item Is it possible that $K(f_c)=\emptyset$?
        \begin{itemize}
            \item No, it is not --- as a degree 2 polynomial, $f_c-z$ has at least one root, which will by necessity be a fixed point, i.e., not diverge to infinity, i.e., in the filled Julia set.
        \end{itemize}
        \item Could it be a finite set?
        \begin{itemize}
            \item No --- $K(f_c)$ is a \textbf{perfect set}.
            \item Uncountably infinite, too.
        \end{itemize}
        \item Is $K(f_c)$ connected?
        \begin{itemize}
            \item Sometimes.
        \end{itemize}
    \end{itemize}
    \item \textbf{Perfect set}: A set where every point in the set is a \textbf{nontrivial limit point} of the set.
    \begin{itemize}
        \item Example: A closed interval, \emph{others listed}.
    \end{itemize}
    \item \textbf{Nontrivial limit point}: A point $p$ in a set $A$ such that there is a nontrivial sequence (i.e., not a constant sequence, e.g., $p,p,p,\dots$) of points in $A$ that converge to $p$.
    \item \textbf{Not connected}: A set $X\subset\C$ such that there exist disjoint, open sets $U,V$ such that $X\subset U\cup V$, $X\cap U\neq\emptyset$, and $X\cap V\neq\emptyset$.
    \item \textbf{Mandlebrot set}: The set of complex numbers $c\in\C$ such that $K(f_c)$ is connected \emph{Also known as} $\bm{M}$.
    \item We can prove that $K(f_c)$ is connected if and only if the critical point of $f_c$ is an element of $K(f_c)$.
    \begin{itemize}
        \item Remember that critical points of $f_c$ are equivalent to zeroes of $f_c'$.
    \end{itemize}
    \item Note that critical points of $f_c:=z^2+c$ are equal to the roots of $f_c'=2z$, i.e., the elements of $\{0\}$.
    \item $K(f_c)$ is connected is equivalent to the sequence $0\to c\to c^2+c\to (c^2+c)^2+c\to\cdots$ is bounded (an absolute value).
    \begin{itemize}
        \item Thus, $c\in M$ is equivalent to the sequence $0\to c\to c^2+c\to (c^2+c)^2+c\to\cdots$ is bounded.
    \end{itemize}
    \item The Mandelbrot set is compact, too.
    \item Proposition: $K(f_c)$ is connected if and only if $0\in K(f_c)$.
    \begin{itemize}
        \item "Proof": $\C-K(f_c)=\bigcup_n{f_c}^{-n}(U)$ where $U$ is an open neighborhood of $\infty$, i.e., the set $\{z:|z|>T\}$.
        \item Let $X_n:=\C-{f_c}^{-n}(U)$, i.e., $X_0=\C-U$, so $K(f_c)=\bigcap_nX_n$.
        \item Cyclic map? $X_n$ getting "smaller" as $n$ increases? $X_{n+1}\subset X_n$.
        \item Assume $X_n=\text{little}$.
        \item Two cases: $X_n$ contains 0 and $X_n$ does not contain 0.
        \item Either every preimage of $X_n$ is connected or there is a $T$ such that for all $n\geq T$, $X_n$ is not connected.
    \end{itemize}
    \item Theorem (Douady-Hubbard): $M$ is connected.
\end{itemize}




\end{document}