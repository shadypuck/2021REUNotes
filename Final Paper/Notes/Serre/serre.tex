\documentclass[../notes.tex]{subfiles}

\pagestyle{main}
\renewcommand{\chaptermark}[1]{\markboth{\chaptername\ \thechapter\ (#1)}{}}
\setcounter{part}{1}

\begin{document}




\part{Serre}
\chapter{Generalities on Linear Representations}
\section{Definitions}
\begin{itemize}
    \item \marginnote{7/15:}$\bm{GL(V)}$: The group of isomorphisms of $V$ onto itself, where $V$ is a vector space over the field $\C$ of complex numbers.
    \item If $(e_i)$ is a finite basis of $n$ elements for $V$, then each linear map $a:V\to V$ is defined by a square matrix $(a_{ij})$ of order $n$.
    \item The coefficients $a_{ij}$ are complex number derived from expressing the images $a(e_j)$ in terms of the basis $(e_i)$ and solving, i.e., we know that each $a(e_j)=\sum_ia_{ij}e_i$.
    \item $a$ is an isomorphism $\Longleftrightarrow$ $\det(a)=\det(a_{ij})\neq 0$.
    \item We can thus identify $GL(V)$ with the group of invertible square matrices of order $n$.
    \item \textbf{Linear representation} (of $G$ in $V$): A homomorphism $\rho:G\to GL(V)$, where $G$ is a finite group.
    \item \textbf{Representation space} (of $G$): The vector space $V$, given a homomorphism $\rho$. \emph{Also known as} \textbf{representation}.
    \item \textbf{Degree} (of a representation $V$): The dimension of the representation space.
    \item \textbf{Similar} (representations): Two representations $\rho,\rho'$ of the same group $G$ in vector spaces $V$ and $V'$ such that there exists a linear isomorphism $\tau:V\to V'$ which satisfies the identity $\tau\circ\rho(s)=\rho'(s)\circ\tau$ for all $s\in G$. \emph{Also known as} \textbf{isomorphic}.
    \begin{itemize}
        \item When $\rho(s),\rho'(s)$ are given in matrix form by $R_s,R_s'$, respectively, this means that there exists an invertible matrix $T$ such that $T\cdot R_s=R_s'\cdot T$ for all $s\in G$.
    \end{itemize}
\end{itemize}


\section{Basic Examples}
\begin{itemize}
    \item \textbf{Unit representation}: The representation $\rho$ of $G$ defined by $\rho(s)=1$ for all $s\in G$. \emph{Also known as} \textbf{trivial representation}.
    \item \textbf{Regular representation}: The representation $\rho$ defined by $\rho(s)=f:V\to V$ where $f:e_t\mapsto e_{st}$, where $V$ is a vector space of dimension $g=|G|$ with basis $(e_t)_{t\in G}$.
    \item \textbf{Permutation representation}: The representation $\rho$ defined by $\rho(s)=f:V\to V$, where $f:e_x\to e_{sx}$, $x\in X$ being the set acted upon by $G$.
\end{itemize}


\begin{enumerate}
    \item So $\rho$ is the representation of a group, technically. But it seems like we more often treat $V$ as the representation. So which is it, because it seems like they are distinct concepts?
    \item What is the utility of representation theory in mathematics? Does mapping group elements onto automorphisms that obey similar properties in the more well-defined vector space allow us to prove certain results about groups, for instance? Is it the other way around, in that representation theory allows us to prove results about vector spaces from what we know about groups? Is representation theory purely a way of linking group theory and linear algebra so that results in one field may be applied to the other and vice versa? I'm just trying to wrap my head around the motivation for creating and studying homomorphism from groups to vector space automorphisms/linear transformations.
    \item Subrepresentations and blocks of block-diagonal matrices?
    \item Can you explain kernel to me?
    \item What about stability under subgroups of G?
    \item Equivalence between representation and group actions?
    \item Are linear automorphisms a generalization of permutations?
    \item It seems like the general linear group is a group, and all that representation theory does is maps an arbitrary group onto a general linear group in a manner that preserves the group operation. But why bother? If we wanted to study the group-like characteristics of the general linear group, couldn't we just do that directly? Or is the point to have a common reference point for a whole bunch of groups?
    \item We define a permutation to be a function because the "original set" and the "final set" are both critical to understanding the nature of what a permutation is. We do the same for a representation for the same reason?
\end{enumerate}




\end{document}