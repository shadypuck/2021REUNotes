\documentclass[../main.tex]{subfiles}

\pagestyle{main}
\renewcommand{\chaptermark}[1]{\markboth{\chaptername\ \thechapter}{}}

\begin{document}




\chapter{}
\section{Gaussian Curvature (Neves)}
\begin{itemize}
    \item \marginnote{6/28:}Plan:
    \begin{enumerate}
        \item What is a surface?
        \item What is the tangent space?
        \item What are the principal curvatures?
        \item What is the Gaussian curvature?
    \end{enumerate}
    \item In analysis:
    \begin{enumerate}
        \item What is a function?
        \item What is the derivative?
        \item What is the Hemian of function?
        \item 2nd derivative test (determinant of Hemina).
    \end{enumerate}
    \item \textbf{Surface}: A subset $\sigma\subseteq\R^3$ such that for all $p\in\Sigma$, there's a neighborhood $B$ of $p$ in $\R^3$ so that $\Sigma\cap B$ "looks like a disk." More precisely, there exists an open neighborhood $U\subseteq\R^2$ and a map $\varphi:U\to\Sigma\cap B\subseteq\R^3$ such that
    \begin{enumerate}[label={\roman*)}]
        \item $\varphi$ is continuous and smooth.
        \item $\varphi$ is a bijection (with $\varphi^{-1}$ continuous).
        \item $\dd{\varphi}_{|x}:\R^2\to\R^3$ is injective for all $x\in U$.
    \end{enumerate}
    \item \textbf{Chart}: The quantity $(\varphi,U)$ near $p\in U$.
    \item Examples:
    \begin{enumerate}[label={\Alph*)}]
        \item Plane. $\Sigma=\R^2\times|0|\subseteq\R^3$ is a surface with chart $\varphi:\R^2\to\Sigma$ where $(x,y)\mapsto(x,y,0)$.
        \item Sphere. $\Sigma=\{\vec{u}\in\R^3\mid|\vec{u}|=1\}$.
        \begin{itemize}
            \item Charts: Consider the sets $U=\{(x_1,x_2)\mid {x_1}^2+{x_2}^2<1\}\subseteq\R^2$. Let $\varphi_1^+:U\to\Sigma\cap\{(x,y,z)\mid x>0\}$ be defined by $\varphi_1^+(u_1,u_2)=(\sqrt{1-{x_1}^2-{x_2}^2},u_1,u_2)$, $\varphi_1^-:U\to\Sigma\cap\{(x,y,z)\mid x<0\}$ be defined by $\varphi_1^-(u_1,u_2)=(-\sqrt{1-{x_1}^2-{x_2}^2},u_1,u_2)$.
            \item Same thing for $\varphi_2^\pm,\varphi_3^\pm$.
        \end{itemize}
        \item A cone $\Sigma=\{(x,y,z)\mid z=\sqrt{x^2+y^2}\}$ is \emph{not} a surface because it fails property (iii).
        \item The closed unit disk $\Sigma=\{(x,y,0)\mid x^2+y^2\leq 1\}$ is also not a surface.
    \end{enumerate}
    \item \textbf{Tangent space}: Let $\Sigma\subseteq\R^3$ be a surface and let $p\in\Sigma$. Then $T_p\Sigma\subseteq\R^3$ is the $z$-plane so that $p+T_p\Sigma$ is the affine plane that best approximates $\Sigma$ near $p$.
    \begin{itemize}
        \item Best linear approximation near the surface.
        \item Very similar/analogous to the derivative.
    \end{itemize}
    \item If $(\varphi,U)$ is a chart near $p$, then $T_p\Sigma=\text{span}\,\{\frac{\partial\varphi}{\partial x_1}(\bar{x}_1,\bar{x}_2),\frac{\partial\varphi}{\partial x_2}(\bar{x}_1,\bar{x}_2)\}$.
    \item Proves linear independence of above vectors.
    \item Principal curvatures.
    \item Let $\Sigma\subseteq\R^3$ be a surface, $p\in\Sigma$, and $\vec{N}$ be a unit normal vector defined around $p$ (i.e., $\vec{N}(q)\cdot\vec{v}=0$ for all $q$ near $p$ and $\vec{v}\in T_q\Sigma$).
    \item Choose $\vec{v}\in T_p\Sigma$ such that $|\vec{v}|=1$. Set $P_v=\text{span}\,\{\vec{v},\vec{N}(p)\}$. Claim: $(\Sigma-p)\cap P_v$ is a curve near the origin.
    \item \textbf{Principal curvature}: The reciprocal of the radius of the circle in $P_v$ that best approximates $(\Sigma-p)\cap P_v$ near the origin. \emph{Also known as} $\bm{K(\vec{v})}$.
    \begin{itemize}
        \item The sign is positive if the center of the circle is in the direction of $\vec{N}(p)$ and negative otherwise.
        \item If the sign of $\vec{N}(p)$ changes, then $K(\vec{v})$ will change in sign.
    \end{itemize}
    \item If we change $\vec{N}$ by $-\vec{N}$, then the new $K(\vec{v})$ is the opposite of the old one.
    \item Given $p\in\Sigma$ and $\vec{N}(p)$ a normal vector at $p$, we define $K_1(p)$ to have the maximum $K(\vec{v})$ over all unit vectors $\vec{v}\in T_p\Sigma$ and $K_2(p)=\min\{K(\vec{v})\mid \vec{v}\in T_p\Sigma,|\vec{v}|=1\}$.
    \item $K_1,K_2$ are computable quantities.
\end{itemize}



\section{Lecture 1.6: An Explicit Formula for the Catalan Numbers}
\begin{itemize}
    \item \marginnote{6/29:}Theorem (discovered by Euler):
    \begin{equation*}
        C_n = \frac{1}{n+1}\binom{2n}{n}=\frac{(2n)!}{n!(n+1)!}
    \end{equation*}
    \item Examples:
    \begin{itemize}
        \item $C_1=\frac{1}{2}\binom{2}{1}=1$.
        \item $C_2=\frac{1}{3}\binom{4}{2}=2$.
        \item $C_3=\frac{1}{4}\binom{6}{3}=5$.
    \end{itemize}
    \item Dyck path:
    \begin{itemize}
        \item We'll study paths starting with $(0,0)$, going on each stop either from $(x,y)$ to $(x+1,y+1)$ or from $(x,y)$ to $(x+1,y-1)$.
        \item Consider the number of ways to get to each point on the integer grid $\Z\times\Z$ from $(0,0)$.
        \begin{itemize}
            \item Generates a rotated Pascal's triangle.
            \item The number of paths from $(0,0)$ to $(a+b,a-b)$, i.e., $a$ moves up and $b$ moves down is $\binom{a+b}{b}$.
        \end{itemize}
    \end{itemize}
    \item Proposition: $C_n$ is equal to the number of paths from $(0,0)$ to $(2n,0)$ which are contained in the upper half-plane ($y\geq 0$).
    \begin{itemize}
        \item Proof: $C_n$ is the number of sequences of brackets.
        \item Transform a sequence of brackets into a path by $(\mapsto\nearrow$ and $)\mapsto\searrow$.
        \item The condition $\#(=\#)$ implies that the paths start at $(0,0)$ and end at $(2n,0)$.
        \item The condition that in every initial segment, $\#(=\#)$ implies that the path lies in the upper half plane.
    \end{itemize}
    \item Reflection principle:
    \begin{itemize}
        \item The number of paths from $A$ to $B$ in the upper half plane is equal to the number of paths from $A$ to $B$ minus the number of paths from $A$ to $B$ that intersect the line $y=-1$.
        \item Symbolically, $C_n=\binom{2n}{n}-$?
        \item There exists a one-to-one correspondence between two sets: The set of all paths from $A$ to $B$ intersecting $\ell$ and the set of all paths from $A$ to $B'$, where $B'$ is the reflection of $B$ across $\ell$.
    \end{itemize}
    \item Thus, the number of paths from $A$ to $B$ that intersect the line $y=-1$ is equal to the number of paths from $A$ to $(2n,-2)$.
    \item Therefore,
    \begin{align*}
        \binom{2n}{n}-\binom{2n}{n-1} &= \frac{(2n)!}{n!n!}\left( 1-\frac{n}{n+1} \right)\binom{2n}{n-1}\\
        &= \frac{1}{n+1}\binom{2n}{n}
    \end{align*}
    \begin{itemize}
        \item Note that it's not obvious that $\frac{1}{n+1}\binom{2n}{n}$ is an integer unless you present it as the difference of two binomials (i.e., as $\binom{2n}{n}-\binom{2n}{n-1}$).
    \end{itemize}
    \item Exercise:
    \begin{itemize}
        \item Take a path from $A$ to $B$ intersecting $\ell$ and find the closest point of $P\cap\ell$ to $B$. Reflect the segment of the path after this point.
    \end{itemize}
\end{itemize}




\end{document}